\documentclass{article}
\usepackage[utf8]{inputenc}
\usepackage{amsmath}
\usepackage[margin=2px]{geometry}
\title{Data Fundamentals}
\author{}
\date{}

\begin{document}

\small

\subsection*{Arrays}

\noindent \textbf{Basic Concepts}
2 $\times$ 3 means 2 rows, 3 columns.
Rank 1 tensor is 1D array.
concatenate to join along existing dimension.
stack to stack up arrays along new dimension.
Tiling to repeat an array.

\noindent \textbf{Broadcasting}
Broadcasting (x, y) with (x, y), (1, y), (x, 1), (y,).

\noindent \textbf{Reduction and Accumulation}
Reduction: apply a function to reduce the array to a single value.
or with axis to reduce along a specific axis.
Accumulate: apply a function to reduce the array to a single value.

% TODO: Probably can remove this section, or at least reduce it.
\noindent \textbf{NumPy Functions}
\begin{itemize}
    \item \texttt{np.loadtxt}, \texttt{np.savetxt}, \texttt{np.zeros}, \texttt{np.ones}, \texttt{np.full}, \texttt{np.empty}, \texttt{np.zeros\_like}, \texttt{np.ones\_like}, \texttt{np.full\_like}, \texttt{np.empty\_like}, \texttt{np.arange}, \texttt{np.linspace}, \texttt{np.array}, \texttt{np.meshgrid}
    \item \texttt{np.random.uniform}, \texttt{np.random.normal}, \texttt{np.random.randint}, \texttt{np.random.choice}, \texttt{np.random.permutation}
    \item \texttt{np.tile}, \texttt{np.transpose}, \texttt{x.T}, \texttt{np.stack}, \texttt{np.concatenate}, \texttt{np.squeeze}, \texttt{np.reshape}, \texttt{np.einsum}, \texttt{np.ravel}, \texttt{np.swapaxes}, \texttt{np.rollaxes}
    \item \texttt{np.minimum}, \texttt{np.maximum}, \texttt{np.add}, \texttt{np.subtract}, \texttt{np.multiply}, \texttt{np.log}, \texttt{np.exp}, \texttt{np.sin}, \texttt{np.cos}, \texttt{np.tan}, \texttt{np.arcsin}, \texttt{np.arccos}, \texttt{np.tanh}
\end{itemize}


\subsection*{Floats}

\noindent \textbf{Float Exceptions}
Float exceptions:
Invalid operation, divide by zero, overflow, underflow, inexact.
np.allclose.

\noindent \textbf{Stride and Representation}
Stride: number of bytes between each element in an axis.
float = sign * 2\^{exponent} * 1.mantissa
dope vector:
numpy arrays hold dopy vectors: stiding information.

\noindent \textbf{Rank and Dimensions}
Rank preserving: The number of dimensions is the same.
Rank reducing: The number of dimensions is reduced.
Rank promoting: The number of dimensions is increased.
Add singleton dimensions: x[:, np.newaxis]
Remove singleton dimensions: np.squeeze (x)
Elided axes: [0, \ldots, 4]
Swapping and rearranging: np.swapaxes, np.rollaxis, np.moveaxis, np.transpose
Einsum: Einstein summation convention. Used to reorder high-dimensional arrays.

\subsection*{Scientific Visualisation}

\noindent \textbf{Basic Terminology}
A \textbf{stat} computes statistics from data, such as means (mean, median, max, min,  etc.)
A \textbf{mapping} transforms data attributes into visual values.
A \textbf{scale} specifies how units are transformed.
A \textbf{coord} system connects mapped data onto points on a plane.
A \textbf{guide} provides visual references like tick marks, labels, and legends to explain the mapping's meaning.
A \textbf{geom} is the geometric representation of mapped data.
A \textbf{layer} consists of one set of geoms with one mapping on one coordinate system, and multiple layers can be overlaid.
A \textbf{facet} shows a different view of the same dataset on a separate coordinate system.
A \textbf{figure} is a collection of one or more facets. Finally.
A \textbf{caption} explains the visualization to the reader.

\noindent Avoid rescaled units. Facet: different view of the same dataset.
Regression: fit a line to the data. Smoothing: fit a curve to the data.

\noindent \textbf{Geoms and Aesthetics}
\noindent Geoms: markers: geoms that represent bare points.
Colour changes: percaptually uniform, monotonic brightness.

\noindent Geometric representations, or \textbf{geoms}, that connect points together should be used if it makes sense to ask what is between two data points.
\textbf{Line styles} can have variable thickness, variable color, and dash patterns to enhance the visual representation of the data.

\noindent For example, the staircase plot is useful when we know that the value cannot have changed between measurements (e.g., in a coin toss scenario).
This type of plot connects points but keeps the value fixed until a new data point is observed.
Conversely, if measurements are naturally discrete, a bar chart may be more suitable to represent the data effectively.

\noindent \textbf{Transparency (Alpha)}
\noindent A \textbf{geom} can be rendered with different levels of transparency,
referred to as \textbf{alpha} (equivalent to opacity) or
\textbf{transparency} (the inverse of opacity).
This feature is particularly useful when dealing with a large number
of overlapping geoms, as it allows for the emphasis of certain geoms
while maintaining visibility of others.
However, it is important to use transparency judiciously, as excessive
transparency can make graphs difficult to read.

\noindent \textbf{Axes and Coordinates}
Axis limits specify a range in data units which are then mapped onto the available space in the figure in visual units.
log scales: semilog, loglog. Symmetric log scales: logit, log.
polar coordinates: useful for circular data.

\noindent \textbf{Facets and Layers}
acets and layers: ways of crating graphs with multiple geoms.
\textbf{Distinct layers} superimposed on the same set of coords.
\textbf{Distinct facets} on separate sets of coords.

\noindent \textbf{Communicating Uncertainty}


\subsection*{Linear Algebra}

\noindent \textbf{Weighted Sums of Vectors}
$\lambda_1 \mathbf{x}_1 + \lambda_2 \mathbf{x}_2 + \cdots + \lambda_n \mathbf{x}_n$
\textbf{Linear interpolation}
$lerp(x_1, x_2, \alpha) = (1 - \alpha)x_1 + (\alpha)x_2$

\noindent \textbf{Norms}
$||x||_p = {(\sum_{i=1}^{n} |x_i|^p)}^{\frac{1}{p}}$
$||x||_\infty = \max_{i=1}^{n} |x_i|$
$||x||_{-\infty} = \min_{i=1}^{n} |x_i|$
\textbf{Normalisation} $x' = \frac{x}{||x||_p}$

\noindent \textbf{Cosine Distance}
$\cos \theta = \frac{\mathbf{x} \cdot \mathbf{y}}{||\mathbf{x}|| \, ||\mathbf{y}||}$

\noindent \textbf{Variance}
$\sigma^2 = \frac{1}{N - 1} \sum_{i=0}^{N-1} {(x_i - \mu)}^2$

\noindent \textbf{Covariance Matrices}.
We compute the covariance of every dimension with every other dimension.
$\Sigma_{ij} = \frac{1}{N - 1} \sum_{k=1}^{N} (X_{ki} - \mu_i)(X_{kj} - \mu_j)$.
You can also use $np.cov$.







\end{document}
